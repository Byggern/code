\documentclass{article}
\usepackage[utf8]{inputenc}
\usepackage{booktabs}
\title{ByggernDok}
\author{torjehoa }
\date{September 2016}

\begin{document}

\maketitle

\section{SRAM}

The following table is from a bad chip select understanding. CE is the chip select. 


\begin{table}[hb]
\begin{tabular}{cccc|ccp{4cm}}
\multicolumn{4}{c}{Inputs} &\multicolumn{2}{c}{Outputs}&\\
MSB$_0$ & MSB$_1$ & Write & Read & Write & Read&\\\midrule
1 & 0 & 0 & 0 & 0 & 0 &addressing without mode\\ 
1 & 0 & 0 & 1 & 0 & 1 &pass-through read\\
1 & 0 & 1 & 0 & 1 & 0 &pass-through write\\
1 & 0 & 1 & 1 & 0 & 0 &Should not happen\\
\end{tabular}
\caption{Addressing RAM: With 10 as the most dignificant bits we address RAM. When we have matching signals we pass through the read and write signals. }
\end{table}

\subsection{Chip enable}

With the chip enable as an output from the GAL the table is simplified to the same size as the ADC. 

\clearpage

\section{OLED}
We need two bits from the address to figure out if we are talking about the OLED, another bit is needed to find if we are talking about command or data. 
\begin{table}[hb]
\begin{tabular}{ccc|cc}
\multicolumn{3}{c}{Inputs}&\multicolumn{2}{c}{Outputs}\\
MSB$_0$&MSB$_1$&MSB$_2$&Command&Data\\\midrule
0&0&0&1&0\\
0&0&1&0&1\\
\end{tabular}
\caption{00 is the signature for addressing the OLED, a third bit is used to distinguish between command and data. }
\end{table}

\section{ADC}
The table for the ADC is trivial, the first two bits select it and then we will read. 
\begin{table}[hb]
\begin{tabular}{cc|c}
\multicolumn{2}{c}{Inputs} & Output\\
MSB$_0$& MSB$_{1}$& ADC select\\\midrule
0&1&1  \\
\end{tabular}
\caption{The ADC is selected by the sequence 01 and can then be read. }
\end{table}
\end{document}

